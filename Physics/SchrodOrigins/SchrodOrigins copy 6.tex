\documentclass[aps,apl,twocolumn,groupedaddress]{revtex4-1}
%\documentclass[12pt]{article}   	% use "amsart" instead of "article" for AMSLaTeX format

\usepackage{graphicx}				% Use pdf, png, jpg, or eps� with pdflatex; use eps in DVI mode
								% TeX will automatically convert eps --> pdf in pdflatex		
%\usepackage{caption}

\begin{document}

\title{A Novel Route to the Semi-Classic Limit of Quantum Mechanics}
\author{Chris Richardson, Peter Schlagheck, John Martin, ?Nicolas Vandewalle?, Thierry Bastin}
\affiliation{University of Liege}
\date{\today}							% Activate to display a given date or no date

\begin{abstract}

%A wave equation which sheds light on the origin of the Schr\"{o}dinger equation has previously been proposed.  This equation which we call the classical Schr\"{o}dinger-like equation is the Schr\"{o}dinger equation plus an extra non-linear term, a classicality-enforcing potential, which has the effect of canceling out all quantum and wave-like effects.  The Schr\"{o}dinger equation has been recovered from the classical Schr\"{o}dinger-like equation by making assumptions which have the effect of completely canceling the classicality-enforcing potential.
%
%We demonstrate both analytically and numerically that it is not strictly necessary to get rid of the classicality-enforcing potential to recover quantum behavior.  We show that by scaling and not necessarily eliminating the classicality-enforcing potential the linear Schr\"{o}dinger equation is recovered, but with a rescaled $\hbar$.  Even though the classical Schr\"{o}dinger-like equation with a scaled classicality-enforcing potential is non-linear, we demonstrate that it behaves in a very linear way.
%
%We call the classical Schr\"{o}dinger-like equation with a scaled classicality-enforcing potential the transition equation and it can be a tool to explore the transition between the quantum and classical worlds.

We present a novel approach to explore the classical limit of quantum theory.  This approach is based on a non-linear Schr\"{o}dinger equation which includes a classicality enforcing potential.  Scaling of this potential is shown to be equivalent to scaling Plank's constant down to the classical limit.

\end{abstract}

\maketitle

\section{Introduction}

The transition between the classical and quantum worlds has important practical value.  Quantum mechanics is an extremely well tested theory and there is no doubt that all the behavior of classical mechanics is contained entirely within quantum theory.  A notion expressed by the Ehrentfest theorem which says that the expected values of measurement predicted from quantum mechanics obey classical Newtonian motion.  It is however unfeasible to a model macroscopic system without approximating quantum mechanics is some manner.  We therefore develop a new tool that allows description of the smooth transition from the approximate classical regime to the quantum one.



Quantum mechanics is governed by the Schr\"{o}dinger equation[cite]
\begin{eqnarray}
i \hbar \frac{\partial \psi}{\partial t} &=& - \frac{\hbar^2}{2 m} \nabla^2 \psi + V \psi \;, \label{eqn:schrod}
\end{eqnarray}
a linear partial differential equation describes how a quantum state evolves in time.  $\psi$ is a wavefunction whose modulus squared gives a probability density, $m$ is the mass, $V$ is the potential and $\hbar$ is Plank's constant[cite].  This equation leads to behavior that is never observed in classical mechanics such as single particle interference[cite], entanglement[cite] and wave-particle duality[cite].  



Plank's constant in Eq.~\ref{eqn:schrod} is a purely quantum term that does not appear in classical mechanics and so it is reasonable to explore the behavior of the Schr\"{o}dinger equation as $\hbar \rightarrow 0$.  But we can immediately see that the Schr\"{o}dinger equation suffers a reduction in order and looses all usefulness.

Another technique, the WKB approximation~\cite{bib:wkb}, approximates solutions to the Schr\"{o}dinger equation with  a small $\hbar$.  It assumes that the amplitude, $A$, of the wavefunction in its polar form $\psi = A e^{ i S / \hbar }$ where $S$ is the phase, to be slowly varying compared to the phase term $S / \hbar$.  This leads to successful approximate solutions in the semi-classical regime but they become singular near the classical turning point.



Following Madelung~\cite{bib:madelung} and Bohm~\cite{bib:bohm} we can use the polar wavefunction form with the Schr\"{o}dinger and Hamilton-Jacobi equations to derive what Bohm coined the \emph{quantum-mechanical potential},
\begin{eqnarray}
U = - \frac{\hbar^2}{2 m} \frac{ \nabla^2 A}{A} = - \frac{\hbar^2}{2 m} \frac{1}{\left| \psi \right|} \nabla^2 \left| \psi \right| \;.\label{eqn:qm_pot}
\end{eqnarray}
A particle under the influence of this quantum-mechanical potential will exhibit all the unique behavior we associate only with quantum mechanics.  This same term pops up when considering how to describe classical mechanics using the same language as quantum mechanics.

Even though classical mechanics is completely contained in quantum mechanics they are discussed in quite different languages.  To develop a tool that smoothly transitions from the two regimes we must be able to describe classical mechanics using the same language of quantum mechanics, that of wave functions and probabilities.  The equations of classical mechanics predict deterministic trajectories for given initial conditions.  If we have ensemble of initial conditions we also have a similar ensemble of trajectories which defined by $A_{cl}$.  A classical wavefuction, $\psi_{cl} = A_{cl} e^{ i S_{cl} / \hbar }$, can be constructed where $S_{cl}$ is the classical action from the Hamilton-Jacobi equation and $\hbar$ is used to provide a dimensionless argument.  Oriols and Mompart~\cite{bib:obm} use this form of the wavefunction to derive a classical wave equation similar to the Schr\"{o}dinger equation that describes evolution of a classical ensemble of trajectories.  We call it the classical Schr\"{o}dinger-like equation and it is defined as
\begin{eqnarray}
i \hbar \frac{\partial \psi_{cl}}{\partial t} &=& - \frac{\hbar^2}{2 m} \nabla^2 \psi_{cl} \nonumber \\
&+& \left(V + \frac{\hbar^2}{2 m} \frac{1}{\left| \psi_{cl} \right|} \nabla^2 \left| \psi_{cl} \right| \right) \psi_{cl} \;,\label{eqn:class_schrod}
\end{eqnarray}
where the probability density, in analogy to the Schr\"{o}dinger equation, is given from the modulus squared of $\psi_{cl}$.  Eq.~(\ref{eqn:class_schrod}) while being completely classical is similar to the Schr\"{o}dinger equation except for the extra non-linear term which has the effect of canceling out all quantum or wave-like effects.  Schleich et al. refer to this term as the \emph{classicality-enforcing potential}.  It is of course Bohm's quantum-mechanical potential, Eq.~(\ref{eqn:qm_pot}), with the opposite sign.

%An analogy can be made between the ensemble of trajectories $A_{cl}$ and the distribution of probability $A$ and the action $S_{cl}$ and the phase $S$.

We now have a classical theory, governed by the classical Schr\"{o}dinger-like equation, and a quantum theory, governed by the Schr\"{o}dinger equation, that are described using the same language.  We can now explore how to transition between these two theories.

%There are excellent theories to accurately describe the microscopic and the macroscopic worlds, namely quantum and classical mechanics.  However there are systems such as [quandrops] which lie squarely in the middle of the two regimes.  [describe quandrops and its behavior?]  These types of systems incorporate behaviors of both regimes and can transition smoothly from one to the other.  We therefore need tools to describe this transition area between the quantum and the classical regimes.  The development of one such tool is the focus of this paper.
%
%At one end of the transition area is classical mechanics and it obeys Newton's laws~\cite{bib:newton} and the Hamilton-Jacobi equation~\cite{bib:hamiltonjacobi}.  At the other end is quantum mechanics which is governed by the Schr\"{o}dinger equation[cite]
%\begin{eqnarray}
%i \hbar \frac{\partial \psi}{\partial t} &=& - \frac{\hbar^2}{2 m} \nabla^2 \psi + V \psi \;, \label{eqn:schrod}
%\end{eqnarray}
%a linear partial differential equation that describes how a quantum state evolves in time.  $\psi$ is a wavefunction whose modulus squared gives a probability density, $m$ is the mass, $V$ is the potential and $\hbar$ is Plank's constant[cite].  This equation leads to behavior that is never observed in classical mechanics such as single particle interference[cite] and entanglement[cite].  
%
%Before moving away from the two extremes and considering the middle of the transition area it is instructive to describe quantum mechanics in a classical way and vice-versa.  Following Madelung~\cite{bib:madelung} we can write the wavefunction in its polar form, $\psi = A e^{ i S / \hbar }$ where $A$ is the amplitude and $S$ is the phase.  Using this form of the wavefunction with the Schr\"{o}dinger and Hamilton-Jacobi equations Bohm~\cite{bib:bohm} was able to derive what he coined the \emph{quantum-mechanical potential},
%\begin{eqnarray}
%U = - \frac{\hbar^2}{2 m} \frac{ \nabla^2 A}{A} = - \frac{\hbar^2}{2 m} \frac{1}{\left| \psi \right|} \nabla^2 \left| \psi \right| \;.\label{eqn:qm_pot}
%\end{eqnarray}
%
%A classical particle under the influence of this quantum-mechanical potential will exhibit all the unique behavior we associate only with quantum mechanics.  This approach can be reversed to determine what is necessary to make a quantum sate behave in a purely classical manner.  A classical wavefuction, $\psi_{cl} = A_{cl} e^{ i S_{cl} / \hbar }$, analogous to the polar form of the wavefunction can be constructed where $S_{cl}$ is the classical action from the Hamilton-Jacobi equation, $A_{cl}$ is a positive real probabilistic distribution and $\hbar$ is used to provide a dimensionless argument.  Oriols and Mompart~\cite{bib:obm} use this form of the wavefunction to derive a wave equation that describes only classical behavior.  Separately, Schleich et al. \cite{bib:revisited} have also derived the same equation and suggest that it is the origin of the Schr\"{o}dinger equation.  We call it the classical Schr\"{o}dinger-like equation and it is defined as
%\begin{eqnarray}
%i \hbar \frac{\partial \psi_{cl}}{\partial t} &=& - \frac{\hbar^2}{2 m} \nabla^2 \psi_{cl} \nonumber \\
%&+& \left(V + \frac{\hbar^2}{2 m} \frac{1}{\left| \psi_{cl} \right|} \nabla^2 \left| \psi_{cl} \right| \right) \psi_{cl} \;,\label{eqn:class_schrod}
%\end{eqnarray}
%where the probability density, in analogy to the Schr\"{o}dinger equation, is given from the modulus squared of $\psi_{cl}$.  Eq.~(\ref{eqn:class_schrod}) is equivalent to the Schr\"{o}dinger equation plus an extra non-linear term which has the effect of canceling out all quantum or wave-like effects.  Schleich et al. refer to this term as the \emph{classicality-enforcing potential}.  It is of course Bohm's quantum-mechanical potential, Eq.~(\ref{eqn:qm_pot}), with the opposite sign.  The classical Schr\"{o}dinger-like equation can be though of as the Schr\"{o}dinger equation with an extra potential that masks all quantum behavior.
%
%There are several ways to explore the transition between the classical and the quantum.  Since $\hbar$ is a purely quantum term that does not show up in classical mechanics it is reasonable to explore the behavior of the Schr\"{o}dinger equation as $\hbar \rightarrow 0$.  But we can immediately see that the Schr\"{o}dinger equation suffers a reduction in order and looses all usefulness.  Another technique, the WKB approximation~\cite{bib:wkb}, assumes a small $hbar$ by requiring the amplitude $A$ in the polar wavefunction to be slowly varying compared to the phase term $S / \hbar$.  This leads to successful approximate solutions in the semi-classical regime but they become singular near the classical turning point.  We propose a new technique which explores the semi-classical transition area by modifying the classical Schr\"{o}dinger-like equation.
%

Schleich et al. transfer from the non-linear classical Schr\"{o}dinger-like equation, Eq.~(\ref{eqn:class_schrod}),  to the linear Schr\"{o}dinger equation, Eq.~(\ref{eqn:schrod}), by first making the anzatz $\psi_{cl} = \psi$.   They then associate the classicality-enforcing potential with the quantum action.  In this way, the classicality-enforcing potential is canceled out in Eq.~(\ref{eqn:class_schrod}).  We would like to add, however, that it is not strictly necessary to get rid of the classicality-enforcing potential to recover quantum behavior.  We have found that by scaling and not necessarily eliminating the classicality-enforcing potential we can reproduce quantum behavior and recover the linear Schr\"{o}dinger equation with a rescaled $\hbar$.  We do this by inserting a degree of quantumness, $0 \leq \epsilon \leq 1$, into Eq.~(\ref{eqn:class_schrod}) which allows us to explore the smooth transition from the classical world, $\epsilon = 0$, to the quantum one, $\epsilon = 1$.

\begin{eqnarray}
i \hbar \frac{\partial \psi}{\partial t} &=& - \frac{\hbar^2}{2 m} \nabla^2 \psi \nonumber \\ 
&+& \left(V + (1 - \epsilon) \frac{\hbar^2}{2 m} \frac{1}{\left| \psi \right|} \nabla^2 \left| \psi \right| \right) \psi\label{eqn:class_schrod_ep}
\end{eqnarray}

We call this the transition equation and we will show that for all values of $\epsilon \neq 0$ it exhibits quantum behavior.  To demonstrate this we take two approaches.  First in Sec.~\ref{sec:scaling} we show analytically that the transition equation is equivalent to the Schr\"{o}dinger equation with a rescaled $\hbar$.  We then in Sec.~\ref{sec:inter} explore numerically what the effect of scaling the classicality-enforcing potential has on the interference of two Gaussian wave packets.

\section{Equivalence to Scaling Plank's Constant\label{sec:scaling}}

The non-linear transition equation can be shown to be equivalent to the linear Schr\"{o}dinger equation with a $\hbar$ scaled by the degree of quantumness,
\begin{eqnarray}
\tilde{\hbar} = \hbar \sqrt{\epsilon} \;. \label{eqn:hbar_scaled}
\end{eqnarray}

We begin by inserting the polar form of the wave equation into the transition equation, Eq.~(\ref{eqn:class_schrod_ep}), and finding the individual elements.  The Laplacian is found to be
\begin{eqnarray}
\nabla^2 \psi &=& \big[ \nabla^2 A + 2 \frac{i}{\hbar}  (\nabla A) \cdot (\nabla S) \nonumber \\
&+& \frac{i}{\hbar} A \nabla^2 S - \frac{1}{\hbar^2}  A (\nabla S)^2 \big] e^{ i S / \hbar } \;.
\end{eqnarray}

The time derivative of the polar wavefunction is
\begin{eqnarray}
i \hbar \frac{\partial \psi}{\partial t} &=& (i \hbar \frac{\partial A}{\partial t} - A \frac{\partial S}{\partial t}) e^{ i S / \hbar } \;.
\end{eqnarray}

The contribution from the classicality-enforcing potential is
\begin{eqnarray}
\frac{\psi}{\left| \psi \right|} \nabla^2 \psi &=& (\nabla^2 A) e^{ i S / \hbar } \;.
\end{eqnarray}

Gathering the real and imaginary terms we recover two equations.  The first is the standard continuity equation which enforces the conservation of probability
\begin{eqnarray}
\frac{\partial A}{\partial t} &=& - \frac{1}{m} (\nabla A) \cdot (\nabla S) - \frac{1}{2 m} A \nabla^2 S \;, \label{eqn:continuity} \\
&=& i \frac{\hbar}{2 m} \frac{1}{2 A} \nabla \cdot (\psi^* \nabla \psi - \psi \nabla \psi^*) =  - \frac{1}{2 A} \nabla \cdot \vec{j} \;, \\
0 &=& \frac{\partial \rho}{\partial t} + \nabla \cdot \vec{j} \;,
\end{eqnarray}
where $\frac{\partial \rho}{\partial t}  =\frac{\partial  \left| \psi \right|^2}{\partial t} =  2 A \frac{\partial  A}{\partial t}$ and $\vec{j}$ is the probability current.  The second is an equation very similar to the Hamilton-Jacobi equation.  It differs by the addition of the quantum-mechanical potential, Eq.~(\ref{eqn:qm_pot}) , scaled by the degree of quantumness,
\begin{eqnarray}
 \frac{\partial S}{\partial t} &=& - \frac{1}{2 m} (\nabla S)^2 - V + \epsilon \frac{\hbar^2}{2 m} \frac{ \nabla^2 A}{A} \;.
\end{eqnarray}

This equation can be made to have the appearance of the Schr\"{o}dinger equation by making the substitution $\tilde{\hbar} = \hbar \sqrt{\epsilon}$.   Performing this substitution yields
\begin{eqnarray}
\frac{\partial S}{\partial t} &=& - \frac{1}{2 m} (\nabla S)^2 - V + \frac{\tilde{\hbar}^2}{2 m} \frac{ \nabla^2 A}{A} \;,\label{eqn:hamilton-jacobi_rescaled}
\end{eqnarray}
which has had the effect of canceling out the degree of quantumness.  Eqs.~(\ref{eqn:continuity}) and (\ref{eqn:hamilton-jacobi_rescaled}) are now completely equivalent to the Schr\"{o}dinger equation with a rescaled $\hbar$ and we can write
\begin{eqnarray}
\psi &=& A e^{i S / \tilde{\hbar}} \;, \\
i \tilde{\hbar} \frac{\partial \psi}{\partial t}  &=& -  \frac{\tilde{\hbar}^2}{2 m} \nabla^2 \psi + V \psi \;.
\end{eqnarray}
 
\section{The Interference of Two Wave Packets\label{sec:inter}}

Using quantum mechanics we would expect two wave packets to spread with time and then to be represented by the standard Young interference pattern~\cite{bib:Young}.  For classical particles with no wave nature we expect the particles to behave like baseballs and pass through one or the other slit and continue on without diffraction for all time.  We can explore the transition between these two extremes by solving the non-linear transition equation numerically.  For comparison, we first solve the system using the fully quantum Schr\"{o}dinger equation.

\subsection{Fully Quantum Wave Packets}

\begin{figure*}[ht]
\begin{minipage}[t]{0.3\textwidth}
%\centering
  \includegraphics[width=1\textwidth]{Graphics/Probs_ep-10.pdf}
  %\caption*{(a)}
\end{minipage}
\begin{minipage}[t]{0.3\textwidth}
%\centering
  \includegraphics[width=1\textwidth]{Graphics/Probs_ep-98.pdf}
  %\caption*{(b)}
\end{minipage}
\begin{minipage}[t]{0.3\textwidth}
%\centering
  \includegraphics[width=1\textwidth]{Graphics/Probs_ep-95.pdf}
  %\caption*{(c)}
\end{minipage}
\begin{minipage}[t]{0.3\textwidth}
%\centering
  \includegraphics[width=1\textwidth]{Graphics/Probs_ep-8.pdf}
  %\caption*{(d)}
\end{minipage}
\begin{minipage}[t]{0.3\textwidth}
%\centering
  \includegraphics[width=1\textwidth]{Graphics/Probs_ep-4.pdf}
  %\caption*{(e)}
\end{minipage}
\begin{minipage}[t]{0.3\textwidth}
%\centering
  \includegraphics[width=1\textwidth]{Graphics/Probs_ep-0.pdf}
  %\caption*{(f)}
\end{minipage}
\caption{Solid line is the analytic probability density using the Schr\"{o}dinger equation with a scaled $\hbar = \tilde{\hbar} \sqrt{\epsilon}$ and the dashed line is the simulated probability density using the transition equation.  All plots are evaluated at the same time $t = 20 \frac{m \sigma^2}{\hbar}$ and the distance between the center of the two gaussians is $d = 6 \sigma$. Plot (a) is the fully classical case with $\epsilon = 0$. (b) $\epsilon = 0.02$. (c) $\epsilon = 0.05$. (d) $\epsilon = 0.2$. (e) $\epsilon = 0.6$. (f) is for the case $\epsilon = 1$ when $\tilde{\hbar} = \hbar$.}
\label{fig:diffract_movie}
\end{figure*}

We start the quantum analysis with two gaussians in one dimension with $V = 0$ and initial conditions
\begin{eqnarray}
i \hbar \frac{\partial \psi}{\partial t} &=& -\frac{\hbar^2}{2 m} \frac{\partial^2 \psi}{\partial x^2} \;.  \\
\psi(x,0) &=& \sqrt{N_0} \left( e^{-\frac{(x-d)^2}{4 \sigma^2}}+e^{-\frac{(x+d)^2}{4 \sigma ^2}}\right) \label{eqn:double_init}
\end{eqnarray}
where $d$ is the distance from the origin to the centers of the Gaussians and $\sigma$ is the root mean square (rms) width. The normalization is
\begin{eqnarray}
 N_0 = 1 / \left(2 \sqrt{2 \pi } \sigma  \left(e^{-\frac{d^2}{2 \sigma ^2}}+1\right)\right) \;.
 \end{eqnarray}

When the time-dependent Schr\"{o}dinger is solved for the initial condition, Eq.~(\ref{eqn:double_init}), the time-dependent wave function is found to be
\begin{eqnarray}
\psi(x,t) &=& \sqrt{\frac{N_0}{a_t}} \left(e^{-\frac{(x-d)^2}{4 a^2_t}}+e^{-\frac{(x+d)^2}{4 a^2_t}}\right) \;.
\end{eqnarray}
where $a^2_t = 1+i \frac{\hbar}{2 m \sigma^2} t$.  When the modulus is squared the interference term becomes obvious.

\begin{eqnarray}
\left| \psi(x,t) \right|^2 &=& \frac{N_0}{\sigma_t} \Bigg[\left(e^{-\frac{(x-d)^2}{4 \sigma^2_t}}+e^{-\frac{(x+d)^2}{4 \sigma^2_t}}\right)^2 \label{eqn:wf_double_t} \nonumber \\
&-& 4 e^{-\frac{ \left(x^2 + d^2\right)}{2 \sigma^2_t}} \sin^2 \left(\frac{\hbar}{4 m \sigma^2 \sigma^2_t} t x d \right)\Bigg]\end{eqnarray}
where $\sigma^2_t = \frac{\hbar^2}{4 m^2 \sigma^2} t^2 +\sigma^2$ is the time dependent rms width.  This gives the expected Young interference pattern as can be seen in Fig.~(\ref{fig:diffract_movie}-f).

\subsection{The Semi-Quantum Semi-Classical Wave Packets}

To explore the transitional behavior of the interference of two Gaussian wave packets from the quantum to the classical regime we use the transition equation, Eq.~(\ref{eqn:class_schrod_ep}), in one dimension and with $V = 0$
\begin{eqnarray}
i \hbar \frac{\partial \psi}{\partial t} &=& -\frac{\hbar^2}{2 m} \frac{\partial^2 \psi}{\partial x^2} + (1 - \epsilon) \frac{\hbar^2}{2 m} \frac{\psi}{\left| \psi \right|} \frac{\partial^2 \left| \psi \right|}{\partial x^2} \;.  \label{eqn:schrod_class_nov}
\end{eqnarray}

This equation can be solved numerically using the explicit finite difference method.  To do so Eq.~(\ref{eqn:double_init}) is discretized into $\psi_{x_n,0}$ where $x_n = n \Delta x$.  The next time step, $t_n = n \Delta t$, for Eq.~(\ref{eqn:schrod_class_nov}) is given by the recurrence relation
\begin{eqnarray}
\psi(x_n,t_{n+1}) &=& i \frac{\Delta t}{(\Delta x)^2} \Big[ \psi_{x_{n+1},t_n} + \psi_{x_{n-1},t_n} \nonumber \\
&-& \psi_{x_n,t_n} \left(2 + i \frac{(\Delta x)^2}{\Delta t}\right) \nonumber \\
&-& (1 - \epsilon) \frac{\psi_{x_n,t_n}}{\left|\psi_{x_n,t_n}\right|} \Big(\left| \psi_{x_{n+1},t_n} \right| \nonumber \\
&-&  2 \left| \psi_{x_n,t_n} \right| + \left| \psi_{x_{n-1},t_n} \right| \Big)\Big] \;.
\end{eqnarray}

%\begin{figure}[ht]
%\begin{minipage}[t]{0.23\textwidth}
%%\centering
%  \includegraphics[width=1\textwidth]{Graphics/Probs_ep-10.pdf}
%  %\caption*{(a)}
%\end{minipage}
%\begin{minipage}[t]{0.23\textwidth}
%%\centering
%  \includegraphics[width=1\textwidth]{Graphics/Probs_ep-98.pdf}
%  %\caption*{(b)}
%\end{minipage}
%\begin{minipage}[t]{0.23\textwidth}
%%\centering
%  \includegraphics[width=1\textwidth]{Graphics/Probs_ep-95.pdf}
%  %\caption*{(c)}
%\end{minipage}
%\begin{minipage}[t]{0.23\textwidth}
%%\centering
%  \includegraphics[width=1\textwidth]{Graphics/Probs_ep-8.pdf}
%  %\caption*{(d)}
%\end{minipage}
%\begin{minipage}[t]{0.23\textwidth}
%%\centering
%  \includegraphics[width=1\textwidth]{Graphics/Probs_ep-4.pdf}
%  %\caption*{(e)}
%\end{minipage}
%\begin{minipage}[t]{0.23\textwidth}
%%\centering
%  \includegraphics[width=1\textwidth]{Graphics/Probs_ep-0.pdf}
%  %\caption*{(f)}
%\end{minipage}
%\caption{Solid line is the analytic probability density using the Schr\"{o}dinger equation with a scaled $\hbar = \tilde{\hbar} \sqrt{\epsilon}$ and the dashed line is the simulated probability density using the transition equation.  All plots are evaluated at the same time $t = 20 \frac{m \sigma^2}{\hbar}$ and the distance between the center of the two gaussians is $d = 6 \sigma$. Plot (a) is the fully classical case with $\epsilon = 0$. (b) $\epsilon = 0.02$. (c) $\epsilon = 0.05$. (d) $\epsilon = 0.2$. (e) $\epsilon = 0.6$. (f) is for the case $\epsilon = 1$ when $\tilde{\hbar} = \hbar$.}
%\label{fig:diffract_movie}
%\end{figure}

The asymptotic behavior is as expected.  For the completely quantum case, $\epsilon = 1$, the interference pattern that forms, Fig.~(\ref{fig:diffract_movie}-f), is identical to the analytic case, Eq.~(\ref{eqn:wf_double_t}).  For the completely classical case, $\epsilon = 0$, the interference pattern that forms, Fig.~(\ref{fig:diffract_movie}-a), is just that of the initial distribution, Eq.~(\ref{eqn:double_init}).  As can be seen in all the frames of Fig.~(\ref{fig:diffract_movie}) for all values of $\epsilon$ the numerically solved non-linear transition equation is equivalent to the linear Schr\"{o}dinger equation with a scaled $\hbar$
%
%For all values of $0 < \epsilon \leq 1$, given enough time, a far-field diffraction pattern will develop with a visibility of one.  The time for a diffraction pattern to develop increases to infinity as degree of quantumness diminishes, $\epsilon \rightarrow 0$.  The diffraction patterns for the higher values of $\epsilon$ are less developed that for the lower values, but the visibility for all of them is one.

For all values of $0 < \epsilon \leq 1$ an interference pattern develops.  Given enough time the interference pattern will develop into the usual far-field interference pattern with a visibility of one.  As can be seen in Fig.~(\ref{fig:diffract_movie}) the time for a diffraction pattern to develop increases to infinity as degree of quantumness diminishes.  The only value in which no interference is observed is that for $\epsilon = 0$.  When using the transition equation classical mechanics is a special singular case.

\section{Conclusion}

We have demonstrate both analytically and numerically that it is not necessary to get rid of the classicality-enforcing potential to recover quantum behavior.  We have found that by scaling and not necessarily eliminating the classicality-enforcing potential the linear Schr\"{o}dinger equation is recovered, but with a rescaled $\hbar$.  We can scale the degree of quantumness, $\epsilon$ in the transition equation to explore the transition are between the classical and quantum worlds.  This does leave us with a non-linear equation to solve, but it avoids the problems that arise when approximating the Schr\"{o}dinger equation with a small $\hbar$.

It is interesting that pure classical mechanics is only observed for the singular case when the degree of quantumness vanishes completely.  In this sense classical mechanics only occurs at the very limit of quantum mechanics.  It is more normal to have quantum effects than not.  This is evident when observing the microscopic world as it is very difficult to avoid quantum effects.  It is less obvious for macroscopic experiments, where quantum behavior should be absent.  When these experiments are considering in the light of the transition equation it should be noted that quantum behavior is almost never absent and it only needs be teased out.  For experiments that operate in the transition area it is therefore not necessary to make a one-to-one correspondence with quantum mechanics to demonstrate quantum behavior.  The transition equation allows a relaxation of the requirements that an experiment would have abide by to be able to definitively say that quantum behavior was observed.


\begin{acknowledgments}
We would like to thank the [what does ARC stand for?] (ARC) and the rest of the Quandrops team.
\end{acknowledgments}

\begin{thebibliography}{4}

\bibitem{bib:newton}
Newton, Isaac, \emph{Philosophiae Naturalis Principia Mathematica}, London, (1687).

\bibitem{bib:hamiltonjacobi}
Landau, L. D. and Lifshitz, E. M., \emph{Mechanics}, Pergamon Press, Oxford, (1969).

\bibitem{bib:macent}
S. Ghosh, T. F. Rosenbaum, G. Aeppli and S. N. Coppersmith, \emph{Entangled quantum state of magnetic dipoles}, Nature 425 48 (2003).

\bibitem{bib:bohm}
D. Bohm, \emph{A Suggested Interpretation of the Quantum Theory in Terms of ``Hidden'' Variables}. I, Phys. Rev. 85, 166 (1952).

\bibitem{bib:madelung}
E. Madelung, \emph{Quantentheorie in hydrodynamischer}. Z Phys 40:322�326. (1926).

\bibitem{bib:wkb}
Sakurai, J. J., \emph{Modern Quantum Mechanics. Addison-Wesley}, (1993)

\bibitem{bib:obm}
X.Oriols and J.Mompart \emph{Overview of Bohmian Mechanics}pages: 15-147; Chapter 1 of the book \emph{Applied Bohmian Mechanics: From Nanoscale Systems to Cosmology} Editorial Pan Stanford Publishing Pte. Ltd (2012).

\bibitem{bib:revisited}
Wolfgang P. Schleich, Daniel M. Greenberger, Donald H. Kobe, and Marlan O. Scully
\emph{Schr\"{o}dinger equation revisited}
PNAS 2013 110 (14) 5374-5379; published ahead of print March 18, 2013, doi:10.1073/pnas.1302475110

\bibitem{bib:Young}
Thomas Young, emph{Experimental Demonstration of the General Law of the Interference of Light, Philosophical Transactions of the Royal Society of London}, vol 94 (1804).

\bibitem{bib:couder_orbits}
E. Fort, A. Eddi, A. Boudaoud, J. Moukhtar, and Y. Couder, \emph{Path-memory induced quantization of classical orbits}, PNAS 107, 17515 (2010).

\bibitem{bib:theonlymystery}
Feynman, Richard P. \emph{Six Easy Pieces} Reading, MA: Addison-Wesley, 1995.

\end{thebibliography}


\end{document}  